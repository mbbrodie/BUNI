\subsection{Birthday Paradox Test}
\label{sub:Birthday Paradox Test}
%% learning how to use latex

\begin{comment}
Probably needs a little more work.
%% Why
Birthday Paradox Test \cite{arora2018gans} was created to measure whether or not the model had learned the distribution of the training set that the model was trained on. Other metrics are able to rule out if the model has simply memorized the training set but don't answer the question if the model has learned the wanted distribution. 

%% how it works
run the model to generate distribution(or images I think)
(a) Pick a sample Size s 
(b) The Birthday Paradox Text (BPT) looks at the closest(closest by similarity) 20 pairs.
(c) Visually inspect the 20 pairs to see if you be classified as duplicates by humans. (I think in the actual metric)
(d) Repeat a -c
If test says size s has duplicates with reasonable probability then the model has a support space size of s**2
%Problems
Need A human to see if the pairs are in fact similiar?
Each dataset might need a different Heurstic Similarity Measure 
Doesn't work well with blurry images AKA won't work with gans that don't make sharp, realistic images.
\end{comment}

%%actual words
Birthday Paradox Test \cite{arora2018gans} was created to measure whether or not the model had learned the distribution of the training set that the model was trained on.
Other metrics are able to rule out if the model has simply memorized the training set but doesn't answer the question if the model has learned the wanted distribution.

The Birthday Paradox Test is based on the famous \textit{Birthday Paradox} which says that if the model supports a distribution of size N then a sample of $\sqrt{N}$ will most likely have a duplicate in the sample. GANs will almost never produce two identical images so in order to combat this the metric simply says that duplicates are images that have high similarity. The proposed test is the following.
\begin{enumerate}
    \item Pick a sample size s from the generated distribution of the model
    \item Use some measure to choose 20 pairs of the most similar images from the sample.
    \item Have a human look and see if any of the 20 pairs is a duplicate.
    \item Repeat.
\end{enumerate}
If the above test finds a duplicate then the support size is less then $S^2$.

The Birthday Paradox Test has some drawbacks.
The first being that the heuristic needed to find the 20 closest pairs from the sample could be different for each dataset.
Besides just having to have a different way to create pars from each dataset, creating those pars might be difficult.
\cite{arora2018gans} used two different heuristic for finding the 20 most similar pars.
The first being for the CelebA dataset in which they simple used a Euclidean distance in pixel space to determine pars.
But with the CIFAR-10 dataset they had to use a CNN to create the 20 pars.
Another problem caused by the different ways of finding the closest pars is that the found distribution space is an upper bound, the better the heuristic for finding the 20 closest pars the tighter the bound of the estimated distribution versus the actual distribution.
The test also requires a human subject to determine if any of the 20 pairs are considered "duplicates", which means this test can not be completely autonomous. This test as stated in \cite{arora2018gans} also doesn't work well with blurry or non realistic images. 