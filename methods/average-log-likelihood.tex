\subsection{Average Log Likelihood}
\label{sub:Average Log Likelihoodt}

\begin{comment}
% url for explainig what log likelihood is https://www.quora.com/In-generative-models-like-GAN-for-images-they-use-the-log-likelihood-as-a-metric-How-exactly-can-I-compute-that

%  How it works
 Kernel density estimation and Parzen window estimation are used to estimate the density function of a distribution from samples.
 You take samples from your generator and 
 With the estimated density function defined you can use the Kullback-Leibler divergence or similar metrics to calculate the distance between the generated distribution and distribution of the test data.
 
 %problems
 It doesn't work well :D lol
 when the data has high dimensionality a Parzen Window struggles to produce the true log-likelihood of a model.(AKA it is not very accurate)
 If the dimesnsionality is low still requires lots of samples to come close to the true log-likelihood.
 
 Because can't really estimate the log-likelihood very well even if dimensions is small.
 
\end{comment}

 To estimate the density function of the generator, A kernel density estimation or a Parzen window estimation is commonly used.
 The likelihood is then found by taking the test data and measuring its likelihood under the estimated density function.
 But because getting an estimate of likelihood is not possible 
 