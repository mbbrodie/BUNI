\subsection{Frechet Inception Distance}
\label{sub:Frechet Inception Distance}
%% learning how to use latex

\begin{comment}
    how:
        The real and fake data is embedded into the inception network.
        Looks at embedded layer as a Multi-variant-Guassian distribution
        Find mean and covariance for embedded layer
        
    Good: IS can not detect intra-class Mode-Collapse but FID can. (can detect if only makes one image per class) will have bad FID Score
    More robust to noise then IS
    
    Bad:
        Needs Inception Net (that is googles thing?)
        Can not detect over fitting if Memory Gan (A gan that memorizes the training set) will score perfectly with FID.
        Needs a model to classify output.
        
\end{comment}


Frechet Inception Distance addresses one of the fundamental flaws of Inception Score
The flaw being that Inception Score fails to detect intra-class diversity or Mode Collapse \cite{borji2018pros}. 
Frechet Inception Distance works by embedding the sample and generated data into a new feature space by using a specific layer of Inception Net. 
The embedded layer is then viewed as a continuous Gaussian distribution and the mean and covariance of the Gaussian distribution is estimated for the generated data and the training data. 
With the mean and covariance found the Frechet Inception Distance can be calculated by:
\begin{equation}
    FID(x,y) = \| \mu_x - \mu_y \|_2^2 + TR(\Sigma_x + \Sigma_y - 2(\Sigma_x\Sigma_y)^{\frac{1}{2}})
\end{equation}

The Fretchet Inception Distance has a few drawbacks the first being that it requires some model that can classify the training and generated images, such as Inception Net.
Another major flaw is that it can not detect over fitting.
If a model memorizes the training data, the model will receive a perfect FID score \cite{lucic2017gans}.

