%https://pdfs.semanticscholar.org/68e0/4bc0193ed46933247d64f710a75a5d179ca3.pdf
Many statistical Image Quality Assessment (IQA) methods, such Peak Signal to Noise Ratio (PSNR), Mean Squared Error (MSE), Mean Absolute Error (MAE), Maximum Difference (MD), or Average Difference (AD), rely upon the availability of ground truth images \cite{patil2015survey}.
Other IQA methods, such as the structural similarity measure SSIM \cite{wang2004image} or the information theoretic Visual Information Fidelity (VIF) \cite{sheikh2006image}, likewise assume an original image to compare with a test image.
While such methods may be appropriate for evaluating outputs in generative tasks such as image super-resolution \cite{ledig2016photo}, colorization \cite{isola2017image}, or image-to-image transformations \cite{zhu2017unpaired}, these IQA methods are not appropriate for tasks with no ground truth comparison.
Hence, we focus on IQA methods that do not assume the availability of ground truth reference images.