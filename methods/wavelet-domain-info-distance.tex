%wavelet-domain-info-distance.tex
\subsubsection{Wavelet Domain Information Distance}
\label{sub:wavelet_domain_information_distance}
Histograms of wavelet subbands provide an efficient method to represent images.
\cite{wang2005reduced} use a 2-parameter generalized Gaussian density (GDD) model \cite{simoncelli1996noise} to approximate the marginal distribution of wavelet subband coefficients in undistorted reference images.
Specifically, they decompose an image into 12 subbands using a pyramid wavelet transform \cite{simoncelli1992shiftable} at various scales and orientations.
After selecting and generating coefficient histograms for a subset of the generated subbands, this method estimates the parameters of a GDD to represent each subband.

Following the GDD training for reference images, this method calculates the coefficient histograms of the wavelet transform subbands for distorted images.
The distortion level between reference and distored images is calculated as
\begin{equation}
	D = log_2 \Bigg (1 + \frac{1}{D_0} \sum_{k=1}^{K} | \hat d^k(p^k||q^k)| \Bigg )
\end{equation}
where $D_0$ is a scaling constant, $K$ is the number of extracted subbands, $p^k$ is the probability density function (PDD) of the reference images, $q^k$ is the PDD of the distorted images, and $d^k$ is a function that estimates the Kullback-Leibler divergence.

A notable drawback of this method is the need to reduce the number of subbands for efficient computation. 
In addition, wavelets are not shift invariant and cannot capture diagonal features well \cite{kingsbury2001complex}.