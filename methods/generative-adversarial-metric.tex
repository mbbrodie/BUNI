\subsection{Generative Adversarial Metric}
\label{sub:generative_adversarial_metric}

The Generative Adversarial Metric attempts to use a key feature of GANs - the discriminator - to see how good a network has become. 
By training two separate GANs then switching the discriminator for a test set, the metric attempts to obtain a more generalized accuracy reading.
The downside of this metric is that it's not consistent - each discriminator is trained differently, and so the accuracy could vary significantly with each run, even if the generators happened to be very similar.