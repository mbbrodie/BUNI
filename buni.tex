\documentclass[12pt]{article}
\usepackage{amsmath,amssymb,comment}

\title{Believable UNique Index (BUNI): \\A novel metric based on IQA and GAN quantitative evaluation methods}
\date{}
\author{Mike Brodie, Brian Rasmussen, Scott Corbitt, \\ Garrett Lingard, Tony Martinez}
\begin{document}
\maketitle

%%%%%% ADD YOUR GAN METHODS HERE %%%%%%
\section{GAN Quantitative Evaluation Methods}
Generative models, particularly GANs, remain difficult to quantitatively evaluate.
As with traditional metrics, GAN evaluation methods must be efficient and reproducible. 
Unlike other measures, however, an effective GAN metric must also consider mode coverage, output diversity, output quality, and penalize exact memorization of the training data.
%inception-score

Inception Score (IS) \cite{gulrajani2017improved} remains the most popular and widely adopted GAN evaluation metrics.
However, recent theoretical and empirical analyses \cite{barratt2018note, borji2018pros, odena2016conditional} demonstrate that IS does not measure intra-class diversity and fails to detect training set memorization.
Additionally, IS relies on an Inception model pretrained on the 1000-label ImageNet dataset, which may not be appropriate for certain GAN evaluation tasks (e.g. MNIST, CIFAR-10, or other datasets with data and label distributions that significantly differ from ImageNet).

\begin{comment}
See https://arxiv.org/pdf/1802.03446.pdf for ideas. IMPROVE upon their explanations.

If you are absolutely CONFIDENT that the explanation in 1802.03446.pdf is adequate
(i.e. discusses WHY it exists/where it came from, HOW it works, and identifies weaknesses),
commit a .tex file with a 1-line summary/single citation and the words 'SEE 1802.03446.pdf for more details.'

Otherwise, write up the following:
(1) WHY it exists (what preceded it? what problem does it solve)
(2) HOW it works
(3) WEAK points / FAIL cases

Add citations to buni.bib, and commit all your files.
\end{comment}

\subsection{Mode Score} 
\label{sub:mode_score}
MODE score was introduced in \cite{che2016mode} in order to measure Mode Collapse better than Inception Score \cite{gulrajani2017improved}. 
Instead of calculating the KL divergence between the softmax output of each generated image and the overall label distribution, MODE score calculates the KL divergence between the softmax outputs and the label distribution of the \textit{training} data.
MODE Score then subtracts the KL divergence between the overall divergence and the label distribution.
Formally, we describe this as
\begin{equation}
	exp \Big (\mathbb{E}_xKL(\ p(y|x)\ ||\ p(y)\ ) - KL(\ p^*(y)\ ||\ p(y)\ ) \Big ).
\end{equation}

One drawback of this approach is that it does not naturally handle unlabelled datasets.
While \cite{che2016mode} offers an workaround solution using an auxiliary discriminator, this approach does not naturally transfer to other datasets. 
Furthermore, it requires training an additional discriminator, which can be difficult to train stably to convergence.
%%NOTE -> Equation 5 needs expaning
Finally, MODE inherits one of the primary flaws of IS, namely, that it relies upon a pretrained Inception classification network.

%\subsection{Frechet Inception Distance}
\label{sub:Frechet Inception Distance}
%% learning how to use latex

\begin{comment}
    how:
        The real and fake data is embedded into the inception network.
        Looks at embedded layer as a Multi-variant-Guassian distribution
        Find mean and covariance for embedded layer
        
    Good: IS can not detect intra-class Mode-Collapse but FID can. (can detect if only makes one image per class) will have bad FID Score
    More robust to noise then IS
    
    Bad:
        Needs Inception Net (that is googles thing?)
        Can not detect over fitting if Memory Gan (A gan that memorizes the training set) will score perfectly with FID.
        Needs a model to classify output.
        
\end{comment}


Frechet Inception Distance addresses one of the fundamental flaws of Inception Score
The flaw being that Inception Score fails to detect intra-class diversity or Mode Collapse \cite{borji2018pros}. 
Frechet Inception Distance works by embedding the sample and generated data into a new feature space by using a specific layer of Inception Net. 
The embedded layer is then viewed as a continuous Gaussian distribution and the mean and covariance of the Gaussian distribution is estimated for the generated data and the training data. 
With the mean and covariance found the Frechet Inception Distance can be calculated by:
\begin{equation}
    FID(x,y) = \| \mu_x - \mu_y \|_2^2 + TR(\Sigma_x + \Sigma_y - 2(\Sigma_x\Sigma_y)^{\frac{1}{2}})
\end{equation}

The Fretchet Inception Distance has a few drawbacks the first being that it requires some model that can classify the training and generated images, such as Inception Net.
Another major flaw is that it can not detect over fitting.
If a model memorizes the training data, the model will receive a perfect FID score \cite{lucic2017gans}.


%\input{methods/am-score.tex}
%\input{methods/modified-inception-score.tex}
%\subsection{Average Log Likelihood}
\label{sub:Average Log Likelihoodt}

\begin{comment}
% url for explainig what log likelihood is https://www.quora.com/In-generative-models-like-GAN-for-images-they-use-the-log-likelihood-as-a-metric-How-exactly-can-I-compute-that

%  How it works
 Kernel density estimation and Parzen window estimation are used to estimate the density function of a distribution from samples.
 You take samples from your generator and 
 With the estimated density function defined you can use the Kullback-Leibler divergence or similar metrics to calculate the distance between the generated distribution and distribution of the test data.
 
 %problems
 It doesn't work well :D lol
 when the data has high dimensionality a Parzen Window struggles to produce the true log-likelihood of a model.(AKA it is not very accurate)
 If the dimesnsionality is low still requires lots of samples to come close to the true log-likelihood.
 
 Because can't really estimate the log-likelihood very well even if dimensions is small.
 
\end{comment}

 To estimate the density function of the generator, A kernel density estimation or a Parzen window estimation is commonly used.
 The likelihood is then found by taking the test data and measuring its likelihood under the estimated density function.
 But because getting an estimate of likelihood is not possible 
 
%\input{methods/coverage-metric.tex}
%\input{methods/maximum-mean-discrepancy.tex}
\subsection{Birthday Paradox Test}
\label{sub:Birthday Paradox Test}
%% learning how to use latex

\begin{comment}
Probably needs a little more work.
%% Why
Birthday Paradox Test \cite{arora2018gans} was created to measure whether or not the model had learned the distribution of the training set that the model was trained on. Other metrics are able to rule out if the model has simply memorized the training set but don't answer the question if the model has learned the wanted distribution. 

%% how it works
run the model to generate distribution(or images I think)
(a) Pick a sample Size s 
(b) The Birthday Paradox Text (BPT) looks at the closest(closest by similarity) 20 pairs.
(c) Visually inspect the 20 pairs to see if you be classified as duplicates by humans. (I think in the actual metric)
(d) Repeat a -c
If test says size s has duplicates with reasonable probability then the model has a support space size of s**2
%Problems
Need A human to see if the pairs are in fact similiar?
Each dataset might need a different Heurstic Similarity Measure 
Doesn't work well with blurry images AKA won't work with gans that don't make sharp, realistic images.
\end{comment}

%%actual words
Birthday Paradox Test \cite{arora2018gans} was created to measure whether or not the model had learned the distribution of the training set that the model was trained on.
Other metrics are able to rule out if the model has simply memorized the training set but doesn't answer the question if the model has learned the wanted distribution.

The Birthday Paradox Test is based on the famous \textit{Birthday Paradox} which says that if the model supports a distribution of size N then a sample of $\sqrt{N}$ will most likely have a duplicate in the sample. GANs will almost never produce two identical images so in order to combat this the metric simply says that duplicates are images that have high similarity. The proposed test is the following.
\begin{enumerate}
    \item Pick a sample size s from the generated distribution of the model
    \item Use some measure to choose 20 pairs of the most similar images from the sample.
    \item Have a human look and see if any of the 20 pairs is a duplicate.
    \item Repeat.
\end{enumerate}
If the above test finds a duplicate then the support size is less then $S^2$.

The Birthday Paradox Test has some drawbacks.
The first being that the measure needed to find the 20 closest pairs from the sample could be different for each dataset.
Besides just having to have a different way to create pars from each dataset, creating those pars might be difficult.
\cite{arora2018gans} used two different measures for finding the 20 most similar pars.
The first being for the CelebA dataset in which they simple used a Euclidean distance in pixel space to determine pars.
But with the CIFAR-10 dataset they had to use a CNN to create the 20 pars.
And the found distribution space is an upper bound the better the measure of finding the 20 closest pars the tighter the bound on found the distribution versus the actual distribution.
The test also requires a human subject to determine if any of the 20 pairs are considered "duplicates", which means this test can not be completely autonomous. This test as stated in \cite{arora2018gans} also doesn't work well with blurry or non realistic images. 
%\subsection{Classifier Two Sample Test}
\label{sub:classifier_two_sample_test}

Classifier Two Sample tests (C2ST) \cite{borji2018pros} are used to determine if two samples are taken from the same distribution, and according to \cite{2016arXiv161006545L} they have potential in the evaluation of GANs.
A C2ST is performed by creating a binary classifier and training it to discriminate between samples collected from two distributions.
This classifier is then tested on novel samples from these distributions.
If it is unable to accurately classify the samples according to their distribution, then the two distributions are declared to be equal.
A more detailed explanation of the process is found in \cite{borji2018pros} and \cite{2016arXiv161006545L}.
INPUT EQUATION HERE IF WE WANT IT
According to \cite{2016arXiv161006545L}, C2STs are useful because they are easily implemented, can learn data representations dynamically, and are flexible and varied in their implementations.
One of the difficulties of working with C2STs are that they are by nature binary and therefore have difficulty discriminating more than two classes.
Another problem is that C2STs are not well equipped to judge a GAN based on the diversity of its outputs, meaning it can grade very well a network that generates the same image for each input.

SEE IF THERE ARE MORE DIFFICULTIES IN C2STS

%\input{methods/classification-performance.tex}
\subsection{Image Retrieval Performance}
\label{sub:Image Retrieval Performance}

\begin{comment}
  Measures the distributions of distances to the nearest neighbors of some query images (i.e., dive
  Goal is to find Data Set Images that the GAN can't quite reproduce (problem images).
  Convolutional Neural Networks (CNN) are used to give each test and generated image a discriminating vector (the vector used for image retrieval).  This vector is used to determine the distance between
  Nearest neighbors w/ respect to these vectors are obtained.
  2 methods proposed
    1st is to get the jth closest image (usually j=1, so just the closest image) to each image of the test set.  Try it for 2 methods and then compare their median distances.  Smallest median distance wins
    2nd is to get the nearest distance set from the training set to the test set. This is treated as the best distance that model could get. Use the mean distance from this nearest neighbor set, and compare to mean distance from generated set to test set (assume test and generated are of the same size). Use equation

  Pros: Easily see improvement b/w networks.  Easily see if network is performing poorly
  Cons: Requires the training and use of a CNN.  Dependent on given data distribution.  Not guaranteed a wide variety of outputs.  (variety limited by # of pictures in test set).
\end{comment}

%WHY
Wang, et al \cite{wang2016ensembles} state that their image retrieval method of evaluating GANs was implemented because they wanted to measure how GANs and ensembles of GANs represented a data distribution, instead of measureing the quality of the images generated.
%HOW
Their method uses discriminatively trained CNNs to assign each test set image and each generated image an image descriptor, which is then used to determine the nearest neighbor distances between test set images and generated images.  Two methods are provided for analyzing these distances \cite{wang2016ensembles}\cite{borji2018pros}.
\begin{enumerate}
  \item The first method is used to compare two different image generators. Given generator $k$, let $d_{i,j}^k$ be the $j^{th}$ nearest generated image to test image $i$.
  Let $\bf{d}_j^k$ $= \{d_{1,j}^k,\cdots,d_{n,j}^k\}$ be the set of $j^{th}$-nearest distances to all $n$ test images.
  % I slightly copied the wording from wang2016 for the end of that last sentence.  Maybe should revise?
  Note that typically $j=1$ so that only the closest image is used. These distributions are obtained for two generators and the hypothesis that the median of the difference between two nearest distance distributions of generators is zero is tested by the Wilcoxon signed-rank test.
  If the hypothesis passes then the generators are equally good representations of the data distribution.
  Otherwise the better generator can be determined from the test.
  \item The second method is used to determine how well a certain generator performs in relation to an ideal generator. The ideal generator is emulated by the test set seeing as the train and test sets are taken from the same distributions.
   Let $\bf{d}_j^t$ be the distribution of the $j^{th}$ nearest distance between the train and test sets.
   The difference between the ideal distribution $\bf{d}_j^t$ and the distribution from a given generator $\bf{d}_j^k$ is modeled by the relative increase in mean nearest neighbor difference, which is calculated as follows.
\end{enumerate}


\begin{equation}
  \hat{d}_j^k = \frac{\bar{d}_j^k - \bar{d}_j^t}{\bar{d}_j^t}
\end{equation}
\begin{align*}
  \bar{d}_j^k = \frac{1}{N}\sum_{i=1}^Nd_{i,j}^k,j
  &&
  \bar{d}_j^t=\frac{1}{N}\sum_{i=1}^{N}d_{i,j}^t
\end{align*}

%FAIL CASES
Image retrieval performace is useful because it allows one to easily compare two networks or to see if a network is performing poorly.
However, it is not without its faults. It relies on the training and use of a CNN.
It is also dependent upon the given data distribution, and does not guarantee a wide variety of outputs (its variety is limited by the size of the test set).

%\subsection{Generative Adversarial Metric}
\label{sub:generative_adversarial_metric}

The Generative Adversarial Metric attempts to use a key feature of GANs - the discriminator - to see how good a network has become. 
By training two separate GANs then switching the discriminator for a test set, the metric attempts to obtain a more generalized accuracy reading.
The downside of this metric is that it's not consistent - each discriminator is trained differently, and so the accuracy could vary significantly with each run, even if the generators happened to be very similar.
%\begin{comment}
Computes the classification accuracies achieved by these two classifiers on
a validation set, which can be the training set or another set of real images sampled from P(x). If
Pg(x) is close to P(x), we expect to see similar accuracies.
\end{comment}

\subsection{Adversarial Accuracy}
\label{sub:adversarial_accuracy}

Adversarial Accuracy uses human-generated labels to group both the real and generated images.
Then, a classifier tries to determine which group a novel image belongs in. If the generated images are close to the real ones, the classification accuracies will be similar. \cite{papernot2016distillation}

However, this method does not guarantee that the images classified will be human recognizable. 
A classifier could get quite high accuracy on generated images even if the generated images have consistent differences from real images. 
%\subsection{Adversarial Divergence}
\label{sub:adversarial_divergence}

Adversarial Divergence uses human-generated labels to group both the real and generated images.
Then, the distance metric between the real and generated images in each group can be calculated easily. \cite{papernot2016distillation}

This method, however, requires human-generated labels, which can be expensive and therefore less scalable.  % from SAME paper as adversarial-accuracy.tex
%\begin{comment}
Reconstruction error is primarily used in variational autoencoders. It's the error from feature reduction then reconstruction. 

One problem with it is with a generative network, it's hard to estimate exactly how much the encoding is off by, due to the difficulty of guessing what the GAN output "should" have been. Xiang and Li estimated the error using gradient descent on latent code to find a vector that minimizes the L2 norm between the sample generated from the code and the target sample. This makes the evaluation process time consuming. 

\end{comment}

\subsection{Reconstruction Error}
\label{sub:reconstruction_error}

Reconstruction error, a commonly used metric from mathematics to measure the effectiveness of dimensionality reduction, was naturally extended to the world of variational autoencoders. \cite{kingma2013auto}
Reconstruction error measures the L2 norm from a set of test samples. 
The reconstruction error of G on X is defined as: 
\begin{equation}
    \mathcal{L}_{rec} (G,X) = \frac{1}{m}\sum_{i=1}^{m}min_z||G(z)-x^{(i)}||^2
\end{equation}
The downside of this approach is that to estimate the ideal z from the sample x, a costly evaluation process is required. 
\cite{xiang2017effects}
Additionally, this makes the most sense for generative models using some type of encoder, and does not come naturally to most GANs.


%%%%%% ADD YOUR IQA METHODS HERE %%%%%%
\section{Image Quality Assessment Methods}
Many statistical Image Quality Assessment (IQA) methods, such Peak Signal to Noise Ratio (PSNR), Mean Squared Error (MSE), Mean Absolute Error (MAE), Maximum Difference (MD), or Average Difference (AD), rely upon the availability of ground truth images \cite{patil2015survey}.
Other IQA methods, such as the structural similarity measure SSIM \cite{wang2004image} or the information theoretic Visual Information Fidelity (VIF) \cite{sheikh2006image}, likewise assume an original image to compare with a test image.
While such methods may be appropriate for evaluating outputs in generative tasks such as image super-resolution \cite{ledig2016photo}, colorization \cite{isola2017image}, or image-to-image transformations \cite{zhu2017unpaired}, these IQA methods are not appropriate for tasks with no ground truth comparison.
Hence, we focus on IQA methods that do not assume the availability of ground truth reference images.

\subsection{No Reference IQA} % --> NRIQA usually best when distortion type known (i.e. blurring, jpeg lossiness)
\label{sub:no_reference_iqa}
%\input{methods/jpeg-quality-index.tex}
%\input{methods/high-low-frequency-index.tex}
% DIIVINE (Distortion Identification based Image Verify & Integrity Evaluation) PAPER: 'Blind image quality Assessment: From natural scene statistics to perceptual quality'
% BLIINDS-I (Blind Image Integrity notator using DCT Statistics-I) PAPER: 'A DCT statistics- based blind image quality index'
% BLIINDS-II PAPER: 'Image and Video Quality Assessment with BLIINDS-II Algorithm using NSS Approach in DCT domain'
% Natural Image Quality Evaluator

\subsection{Reduced Reference IQA}
\label{sub:reduced_reference_iqa}
%wavelet-domain-info-distance.tex
\subsubsection{Wavelet Domain Information Distance}
\label{sub:wavelet_domain_information_distance}
Histograms of wavelet subbands provide an efficient method to represent images.
\cite{wang2005reduced} use a 2-parameter generalized Gaussian density (GDD) model \cite{simoncelli1996noise} to approximate the marginal distribution of wavelet subband coefficients in undistorted reference images.
Specifically, they decompose an image into 12 subbands using a pyramid wavelet transform \cite{simoncelli1992shiftable} at various scales and orientations.
After selecting and generating coefficient histograms for a subset of the generated subbands, this method estimates the parameters of a GDD to represent each subband.

Following the GDD training for reference images, this method calculates the coefficient histograms of the wavelet transform subbands for distorted images.
The distortion level between reference and distored images is calculated as
\begin{equation}
	D = log_2 \Bigg (1 + \frac{1}{D_0} \sum_{k=1}^{K} | \hat d^k(p^k||q^k)| \Bigg )
\end{equation}
where $D_0$ is a scaling constant, $K$ is the number of extracted subbands, $p^k$ is the probability density function (PDD) of the reference images, $q^k$ is the PDD of the distorted images, and $d^k$ is a function that estimates the Kullback-Leibler divergence.

A notable drawback of this method is the need to reduce the number of subbands for efficient computation. 
In addition, wavelets are not shift invariant and cannot capture diagonal features well \cite{kingsbury2001complex}.

\subsection{Video IQA}
\label{sub:video_iqa}
% Motion-based Video Integrity Evaluation (MOVIE) index


\bibliographystyle{plain}
\bibliography{buni}

\end{document}
